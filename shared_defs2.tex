%\IEEEoverridecommandlockouts
%\overrideIEEEmargins

%floats and figures
%\usepackage{graphics} %imported in the main document.
%\usepackage[pdftex]{graphicx}
\usepackage[font={small}]{caption}
\usepackage{subcaption}
%\usepackage[center]{subfigure} %DONT USE BOTH SUBCAPTION AND SUBFIGURE
%\DeclareGraphicsExtensions{.pdf,.png,.jpg}
%\usepackage{overpic}
\usepackage[rightcaption]{sidecap}
%\usepackage{pbox}

%Math Stuff
\

%DOCUMENT WIDE
\usepackage{times} % assumes new font selection scheme installed
\usepackage{xspace}
%\usepackage[english]{babel} %for hyphenation rules
%\usepackage{flushend}%balance columns on last page
\usepackage{fixltx2e} %fix latex issue across versions
\usepackage{bm}
\usepackage{units}
\usepackage{subfiles} %individual file compilation--makes it quick 
\usepackage{setspace}%line spacing within document

\usepackage{makeidx}
\usepackage{enumitem}
\usepackage[yyyymmdd,hhmmss]{datetime}
\usepackage[english]{babel}

%Bibliography and cross-ref
\makeatletter
\let\NAT@parse\undefined
\makeatother
\usepackage[numbers]{natbib}
\renewcommand{\bibfont}{\footnotesize}
% \usepackage{cite} %DONT USE NATBIB AND CITE TOGETHER

%hyperlinking
\usepackage{url}
\makeatletter
\g@addto@macro{\UrlBreaks}{\UrlOrds}
\makeatother
\usepackage{color}
%\usepackage[usenames,dvipsnames,table,xcdraw]{xcolor}
\usepackage[T1]{fontenc}


%=======U S E R  D E F I N E D  M A C R O S=======
% \newcommand{\bibhref}[2]{#2}
\newcommand{\todo}[1]{\textcolor{red}{[?#1?]}}
\newcommand{\tocite}[1]{\textcolor{red}{[cite]}}
\newcommand{\del}[1]{}
\newcommand{\etal}[1]{et al.}

% Usage:
% \figlabel{myfigure} creates \label{fig:myfigure}
% \figref{myfigure} references it
\newcommand{\figlabel}[1]{\label{fig:#1}}
\newcommand{\figref}[1]{Figure~\ref{fig:#1}}

% Usage:
% \seclabel{mysection} creates \label{sec:mysection}
% \secref{mysection} references it
\newcommand{\seclabel}[1]{\label{sec:#1}}
\newcommand{\secref}[1]{Section~\ref{sec:#1}}

% Usage:
% \tablabel{mytable} creates \label{tab:mytable}
% \tabref{mytable} references it
\newcommand{\tablabel}[1]{\label{tab:#1}}
\newcommand{\tabref}[1]{Table~\ref{tab:#1}}

% use this command instead of writing "da Vinci" so it's never split 
\newcommand{\davinci}{da~Vinci\xspace}

%for table
% \usepackage{floatrow}
% Table float box with bottom caption, box width adjusted to content
% \newfloatcommand{capbtabbox}{table}[][\FBwidth]