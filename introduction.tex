\setcounter{secnumdepth}{0}
\chapter{Introduction}
Bellman's ``Principle of Optimality'' and the characterization of dynamic programming is one of the most important results in computing~\cite{bellman2013dynamic}. Its importance stems from the ubiquity of Markovian decision processes (MDPs), which formalize a wide range of problems from path planning to scheduling~\cite{howard1966dynamic}. In the most abstract form, there is an agent who makes a sequence of decisions to effect change on a system that processes these decisions and updates its internal state possibly in non-deterministically. 
The process is ``Markovian'' in the sense that the system's current state completely determines its future progression. As an example of an MDP, one might have to plan a sequence of motor commands to a robot to control it to a target position. Or, one might have to schedule a sequence of cluster computing tasks while avoiding double scheduling on a node. 
The solution to a MDP is a decision making policy; the optimal decision to make given any current state of the system. 

Since the MDP framework is extremely general---it encompasses shortest-path graph search, supervised machine learning, and optimal control---the difficulty in solving a particular MDP relates to what assumptions can be made about the system.
A setting of interest is when one only assumes black-box access to the system, where no parametric model of the system is available and the agent must iteratively query the system to optimize its decision policy.
This setting, also called \emph{reinforcement learning}~\cite{sutton1998reinforcement}, is interesting for a few reasons.
First, there are many dynamical problems for which a closed-form analytical description of a system's behavior is not available but one has a programatic approximation of how the system transitions (i.e., simulation).
Reinforcement Learning (RL) allows for direct optimization of decision policies over such simulated systems.
Next, by virtue of the minimal assumptions, RL algorithms are extremely general and widely applicable across many different problem settings.
From a Software Engineering perspective, this unification allows the research community to develop a small number of optimized libraries for RL rather each domain designing/maintaining problem-specific algorithms.
Over the last few years, the combination deep neural networks and reinforcement learning, or Deep RL, have emerged as in robotics, AI, and machine learning as an important area of research~\cite{mnih2015human,silver2017mastering, sunderhauf2018limits, stoica2017berkeley}.
Deep RL correlates features of states to successful decisions using neural networks. 

Due to the nature of these black-box queries, RL can be very inefficient in problems where the decision space is high-dimensional and when the decision horizon (the length of the sequence of decisions the agent needs to make) is very long. 
The algorithm has to simultaneously correlate features that are valuable for making a decision in a state while also searching through the space of decisions to figure out which sequence of decisions are valuable. A failure to learn means that the algorithm cannot intelligently search the space of decisions, and a failure to find promising sequences early means that the algorithm cannot learn.
This creates a fundementally unstable algorithm setting, where the hope is the algorithm discovers good decision sequences by chance and can bootstrap from these relatively rare examples.
The higher the dimensionality of the problem, the less likely purely random search will be successful. 

Since Deep RL suffers such limitations, how do practioners solve such large scale ``black-box'' problems today (with non-Deep RL methods)? Every domain has pragmatic approximation algorithms that exploit heuristics, domain knowledge, and other known structure of typical solutions in the problem domain. Consider the following examples:

\begin{example}[SQL Query Optimization]
Most relational database management systems support only dyadic join operations (joining two tables) as primitive operations, a query optimizer must choose the ``best'' sequence of two-way joins to achieve the k-way join of tables requested by a query. The exhaustive decision space is super exponential in k. Since it is common for data to be organized in one large table with several smaller reference tables and typical cost functions are linear in the size of the input tables, most query optimization algorithms restrict their search to left-deep plans.
\end{example}

\begin{example}[Robot Task and Motion Planning]
Planning collision free motions of a robot to meet a control objective is exponential in the time-horizon. Long-term tasks are decomposed into shorter nearly independently solvable planning subtasks. This reduces the effective planning horizon and allows the robot to reuse plans if a subtask is repeated.
\end{example}

Exploiting such structure in Deep RL poses a dilemma---either one sacrifices generality by incorporating problem-specific knowledge into the Deep RL algorithm or one pays the price of exploring an immense search space. Building and maintaining domain-specific Deep RL implementations defeats the unifying premise of Deep RL.

In several domains of interest, a limited amount of expert knowledge is available, e.g., a human teleoperator can guide the motion of a robot to show an example solution of one problem instance.
Collecting such ``demonstrations'' to exhaustively learn a solution for possible progressions on MDP can be very challenging (called imitation learning~\cite{osa2018algorithmic}), but we might have sufficient data to find important structural features of a search problem including task decomposition, subspaces of actions that are irrelevant to the task, and the structure of the cost function.
My thesis is that \emph{learning such structure can significantly improve the sample-efficiency, stability, and robustness of Deep RL in high-dimensional and long-horizon search problems.} 
With this additional supervision, the search process can be restricted to those sequences that are similar to the supervision provided by the expert.
My work over the last 6 years explores this architecture in several Deep RL systems for control of imprecise cable-driven surgical robots, automatically synthesizing data-cleaning programs to meet quality specifications, and generating efficient execution plans for relational queries. I describe algorithmic contributions, theoretical analysis about the implementations themselves, and the architecture of the RL systems.

\vspace{0.75em} \noindent \textbf{Contributions: }  This dissertation contributes:

\begin{enumerate}
    \item Deep Discovery of Options (DDO): A new bayesian learning framework for learning parametrized control hierarchies from expert data (Chapter 2). DDO infers the parameters of an Abstract HMM model to decomposes the action space into a hierarchy of discrete skills relevant to a task. I show that the hierarchical model represents policies more efficiently (requires less data to learn) than a flat model on real and simulated robot control tasks. I also show that this hierarchy can be used to guide exploration in search problems through compositions of the discrete skills seen in the data rather than arbitrary sequences of actions. I apply this model to significantly accelerate learning in self-play of Atari games.

    \emph{Krishnan, Sanjay, Roy Fox, Ion Stoica, and Ken Goldberg. DDDO: Discovery of Deep Continuous Options for Robot Learning from Demonstrations. Proceeding of Machine Learning Research: Conference on Robot Learning. 2017.} 
    
    \item Sequential Windowed Inverse Reinforcement Learning (SWIRL): A learning framework for approximating an MDP with a sequence of shorter horizon MDPs with quadratic cost functions (Chapter 3). SWIRL can be thought of as a special parametric case of DDO where skills terminate at goal states. I show on two surgical robot control tasks, cutting along a line and tensioning fabric, SWIRL significantly reduces the policy search time of a Q-Learning algorithm. I also demonstrate improvements in a robot planning task (parallel parking a car with noisy dynamics) and classical control problems (controlling a two-link pendulum).

    \emph{Krishnan, Sanjay, Animesh Garg, Richard Liaw, Brijen Thananjeyan, Lauren Miller, Florian T. Pokorny, and Ken Goldberg. SWIRL: A Sequential Windowed Inverse Reinforcement Learning Algorithm for Robot Tasks with Delayed Rewards. International Journal of Robotics Research. 2018.}

    \item Transition State Clustering (TSC): The underlying task decomposition algorithm in SWIRL is a Bayesian clustering framework called TSC (Chapter 3). This model correlates spatial and temporal features with changes in motion of the demonstrator. The crucial advantage of this framework is that it can exploit ``third person'' demonstrations data where only the states of the MDP are visible but not the decisions the expert took. The motivating application is learning from expert surgeons by analyzing data from a surgical robot. I show that TSC is more robust to spatial and temporal variation compared to other segmentation methods and can apply to both kinematic and visual demonstration data.

    \emph{Krishnan, Sanjay, Animesh Garg, Sachin Patil, Colin Lea, Gregory Hager, Pieter Abbeel, and Ken Goldberg. Transition State Clustering: Unsupervised surgical trajectory segmentation for robot learning. International Journal of Robotics Research. 2018.}

    \item Alpha Clean: A system for synthesizing data cleaning programs to enforce database integrity constraints (Chapter 4). The main algorithm in the system decomposes the integrity constraints on a table into a collection of independent search problems called blocks. The algorithm starts by exhaustively searching the initial blocks and then incrementally learns a Q-function to make the search on future increasingly precise. This process leverages the fact that data cleaning is not an arbitrary constraint satisfaction problem and most database corruption is systematic, i.e., correlated with features of the data. 

     \emph{Krishnan, Sanjay, Eugene Wu, Michael Franklin, and Ken Goldberg. Alpha Clean: Data Cleaning with Distributed Tree Search and Learning. 2018.}

    \item A new approximate dynamic programming framework for SQL query optimization. Classical query optimizers leverage dynamic programs for optimally nesting join queries. This process creates a table that memoizes cost-to-go estimates of intermediate subplans. By representing the memoization table with a neural network, the optimizer can estimate the cost-to-go of even previously unseen plans allowing for a vast expansion of the search space. I show that this process is a form of Deep Q-Learning where the state is a query graph and the actions are contractions on the query graph. One key result is that the same optimization algorithm can adaptively learn search strategies for both in-memory databases (where the search space is often heuristically restricted based to maximize index accesses) and disk-based databases (where the search space is often heuristically restricted to maximize re-use of intermediate results).

    \emph{Krishnan, Sanjay, Zongheng Yang, Ion Stoica, Joseph Hellerstein, and Ken Goldberg. Learning to Efficiently Enumerate Joins with Deep Reinforcement Learning. 2018.}
    
    \item An application of deep reinforcement learning to synthesizing surgical thin tissue tensioning policies. To improve the search time, the algorithm initializes its search with an analytical approximation of the equilibrium state of a FEM simulator. The result is a search algorithm that searches for a policy over several hundred timesteps in less than a minute of latency (Chapter 6). 

     \emph{Krishnan, Sanjay and Ken Goldberg. Sanjay Krishnan and Ken Goldberg. Bootstrapping Deep Reinforcement Learning of Surgical Tensioning with An Analytic Model. C4 Surgical Workshop 2017.}
    
\end{enumerate}





