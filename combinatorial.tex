\chapter{Reinforcement Learning In Combinatorial Problems} \label{comb}
Consider the following problem. Suppose, we are given a SQL query that integrates information from $k$ different tables.Since relational database management systems typically support only diadic join operators as primitive operations, a query optimizer must choose the ``best'' sequence of two-way joins to achieve the k-way join of
tables requested by a query. 

We can think of this as a sequential decision problem. We start with an initial set of tables $\{t_1,...,t_k\}$. There are $\binom{k}{2}$ possible joins one can make (assuming symmetric joins).
We pick a join and then update the set removing the single tables and replacing them with an intermediate relation signifying the new joined tables $\{t_1,...,t_{ij},...,t_k\}$. This process repeats until there is only a single relation left. The problem of optimally nesting joins is known to be NP-Hard, and we explore whether RL can play a role in efficient approximations.

The basic idea to is going back to the meaning of the Q-Function, namely, it is a function such that defines the optimal cost-to-go for a taking an action in a state. Knowing the Q-Function completely determines the optimal policy since one just needs to select the action that optimizes the Q-function in each state. Another way of stating this definition is that the Q-Function defines a hypothetical perfect ``reward'' for a task, where algorithms can simply greedily choose actions that optimize the Q-function. Applying RL, different approximations for the Q functions learned from data lead to different types of approximate solutions.

While this problem is exactly in the class of sequential decision problems considered in the previous chapters, the combinatorial action space makes it challenging to apply RL algorithms. If each possible join is an action then there are $2^{k}$ possible joins one could ever make (counting joins of intermediate relations). How does one apply classical RL techniques to such problems? Or, should one apply RL techniques to such problems.
We explore this question in two projects automatically synthesizing data cleaning programs, and optimizing the execution of relational queries.

